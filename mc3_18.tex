\documentclass{beamer}

%\usepackage[table]{xcolor}
\mode<presentation> {
  \usetheme{Boadilla}
%  \usetheme{Pittsburgh}
%\usefonttheme[2]{sans}
\renewcommand{\familydefault}{cmss}
%\usepackage{lmodern}
%\usepackage[T1]{fontenc}
%\usepackage{palatino}
%\usepackage{cmbright}
  \setbeamercovered{transparent}
\useinnertheme{rectangles}
}
%\usepackage{normalem}{ulem}
%\usepackage{colortbl, textcomp}
\setbeamercolor{normal text}{fg=black}
\setbeamercolor{structure}{fg= black}
\definecolor{trial}{cmyk}{1,0,0, 0}
\definecolor{trial2}{cmyk}{0.00,0,1, 0}
\definecolor{darkgreen}{rgb}{0,.4, 0.1}
\usepackage{array}
\beamertemplatesolidbackgroundcolor{white}  \setbeamercolor{alerted
text}{fg=red}

\setbeamertemplate{caption}[numbered]\newcounter{mylastframe}

%\usepackage{color}
\usepackage{tikz}
\usetikzlibrary{arrows}
\usepackage{colortbl}
%\usepackage[usenames, dvipsnames]{color}
%\setbeamertemplate{caption}[numbered]\newcounter{mylastframe}c
%\newcolumntype{Y}{\columncolor[cmyk]{0, 0, 1, 0}\raggedright}
%\newcolumntype{C}{\columncolor[cmyk]{1, 0, 0, 0}\raggedright}
%\newcolumntype{G}{\columncolor[rgb]{0, 1, 0}\raggedright}
%\newcolumntype{R}{\columncolor[rgb]{1, 0, 0}\raggedright}

%\begin{beamerboxesrounded}[upper=uppercol,lower=lowercol,shadow=true]{Block}
%$A = B$.
%\end{beamerboxesrounded}}
\renewcommand{\familydefault}{cmss}
%\usepackage[all]{xy}

\usepackage{tikz}
\usepackage{lipsum}

 \newenvironment{changemargin}[3]{%
 \begin{list}{}{%
 \setlength{\topsep}{0pt}%
 \setlength{\leftmargin}{#1}%
 \setlength{\rightmargin}{#2}%
 \setlength{\topmargin}{#3}%
 \setlength{\listparindent}{\parindent}%
 \setlength{\itemindent}{\parindent}%
 \setlength{\parsep}{\parskip}%
 }%
\item[]}{\end{list}}
\usetikzlibrary{arrows}
%\usepackage{palatino}
%\usepackage{eulervm}
\usecolortheme{lily}

\newtheorem{com}{Comment}
\newtheorem{lem} {Lemma}
\newtheorem{prop}{Proposition}
\newtheorem{thm}{Theorem}
\newtheorem{defn}{Definition}
\newtheorem{cor}{Corollary}
\newtheorem{obs}{Observation}
 \numberwithin{equation}{section}


\title[Methodology I] % (optional, nur bei langen Titeln nötig)
{Math Camp}

\author{Justin Grimmer}
\institute[Stanford University]{Professor\\Department of Political Science \\  Stanford University}
\vspace{0.3in}

\date{September 6th, 2018}

\begin{document}



\begin{frame}
\titlepage
\end{frame}


% \begin{frame}


% \only<1>{\scalebox{0.5}{\includegraphics{Rolles1.pdf}}}
% \only<2>{\scalebox{0.5}{\includegraphics{MVT1.pdf}}}


% \end{frame}


% \begin{frame}
% \frametitle{Proving Mean Value Theorem}

% \only<1>{\begin{thm}
% Suppose $f:[a,b]\rightarrow \Re$ is continuous on $[a,b]$ and $f$ is differentiable on $(a,b)$. Then $\exists$ $c \in (a,b)$ such that
% \begin{eqnarray}
% f^{'}(c) & = & \frac{f(b) - f(a) } {b - a} \nonumber 
% \end{eqnarray}

% \end{thm}
% }
% \invisible<1>{\begin{proof}
% Call new function $g(x)  = f(x) - L(x)$.  \\
% Task 1: Define $L(x)$.  To apply Rolle's we'd like $g(b) = g(a) = 0$.  
% \begin{eqnarray}
% L(x) & = & \frac{f(b) - f(a) }{b - a} (x - a) + f(a) \nonumber 
% \end{eqnarray}

% Task 2: Use Rolle's Theorem.  We know that $g(b) = g(a) = 0$.  Now, this implies $\exists$ c $\in (a, b)$ such that 
% \begin{eqnarray}
% g^{'}(c) & =&   0 \nonumber \\
% f^{'}(c) - L^{'}(c) & = & 0 \nonumber \\
% f^{'}(c) - \frac{f(b) - f(a) }{b - a} & = & 0 \nonumber \\
% f^{'} (c) & = & \frac{f(b) - f(a)}{b- a} \nonumber 
% \end{eqnarray}

% \end{proof}
% }
% \pause 
% \end{frame}




\begin{frame}
\frametitle{Optimization} 

Social scientists are often concerned with finding \alert{extrema}: \alert{maxima} or \alert{minima} \pause 
\begin{itemize}
\invisible<1>{\item[-] Given data, \alert{most} likely value of a parameter} \pause 
\invisible<1-2>{\item[-] Game theory: given other player's strategy, action that \alert{maximizes} utility} \pause 
\invisible<1-3>{\item[-] Across substantive areas: what is the \alert{optimal} action, strategy, prediction?} \pause 
\end{itemize}


\invisible<1-4>{\alert{How to Optimize}} \pause 
\begin{itemize}
\invisible<1-5>{\item[-] When functions are \alert{well behaved} and \alert{known} $\leadsto$ analytic solutions} \pause
\begin{itemize}
\invisible<1-6>{\item[-] Differentiate, set equal to zero, solve} \pause
\invisible<1-7>{\item[-] Check end points and use second derivative test} \pause
\end{itemize}
\invisible<1-8>{\item[-] More difficult problems$\leadsto$ computational solutions} \pause
\end{itemize}


\end{frame}

\begin{frame}
\frametitle{Intuition: Optimization with Derivatives, \alert{Known} well behaved functions}

\begin{columns}[]
\column{0.6\textwidth}
\scalebox{0.5}{\includegraphics{rolles3.pdf}}
 \pause 

\column{0.4\textwidth}
\begin{itemize}
\invisible<1>{\item[-] Rolle's theorem guarantee's that, at some point, $f^{'}(x) = 0$ } \pause 
\invisible<1-2>{\item[-] \alert{Intuition from proof}---what happens as we approach from the left?} \pause 
\invisible<1-3>{\item[-] \alert{Intuition from proof}---what happens as we approach from the right?} \pause 
\invisible<1-4>{\item[-] \alert{critical intuition} first, second derivatives}
\end{itemize}
\end{columns}

\end{frame}


\begin{frame}
\frametitle{Second Derivatives}

\begin{defn} Suppose $f:\Re \rightarrow \Re$ is differentiable.  Recall we write this as $f^{'}$ and suppose that $f^{'}:\Re \rightarrow \Re$.  Then if the limit, 
\begin{eqnarray}
\lim_{x\rightarrow x_{0} } R(x) & = & \frac{f^{'}(x) - f^{'}(x_{0} ) } {x - x_{0} } \nonumber 
\end{eqnarray}
exists, we call this the \alert{second derivative} at $x_{0}, f^{''}(x_{0})$.  
\end{defn}

\end{frame}

\begin{frame}
\frametitle{Example of Second Derivatives}

\only<1>{\begin{eqnarray}
f(x) & = & x \nonumber \\
f^{'}(x) & = & 1 \nonumber \\
f^{''} (x) & = & 0 \nonumber 
\end{eqnarray}}


\only<2>{\begin{eqnarray}
f(x) & = & e^{x} \nonumber \\
f^{'}(x) & = & e^{x} \nonumber \\
f^{''}(x) & = & e^{x} \nonumber
\end{eqnarray}}

\only<3>{
\begin{eqnarray}
f(x) & = & \log (x) \nonumber \\
f^{'}(x) & = & \frac{1}{x} \nonumber \\
f^{''}(x) & = & \frac{-1}{x^2} \nonumber 
\end{eqnarray}
}

\only<4>{\begin{eqnarray}
f(x) & = & \frac{1}{x} \nonumber \\
f^{'}(x) & = & \frac{-1}{x^2} \nonumber \\
f^{''}(x) & = & \frac{2}{x^3} \nonumber 
\end{eqnarray}}


\only<5>{\begin{eqnarray}
f(x) & = & -x^2 + 20 \nonumber \\
f^{'}(x) & = & -2 x  \nonumber \\
f^{''}(x) & =& - 2 \nonumber 
\end{eqnarray}}



\end{frame}


\begin{frame}
\frametitle{Approximating functions and second order conditions}


\begin{thm}
\textbf{Taylor's Theorem}
Suppose $f:\Re \rightarrow \Re$, $f(x)$ is infinitely differentiable function.  Then, the taylor expansion of $f(x)$ around \alert{$a$} is given by 

\begin{eqnarray}
f(x) & = & f(a) + \frac{f^{'}(a)}{1!} (x- a) + \frac{f^{''}(a)}{2!} (x - a)^2 + \frac{f^{'''}(a)}{3!}(x- a)^3 + \hdots \nonumber \\
f(x) & = & \sum_{n=0}^{\infty } \frac{f^{n} (a) }{n!} (x - a)^n \nonumber
\end{eqnarray}

\end{thm}


\end{frame}


\begin{frame}

{\tt R Code!}

\end{frame}




\begin{frame}
\frametitle{Concavity, Convexity, Inflections}
Second derivatives provide further information about functions 

\begin{columns}[]

\column{0.5\textwidth}
\only<1>{\scalebox{0.35}{\includegraphics{fig1.pdf}}} 

\only<2>{\scalebox{0.35}{\includegraphics{fig3.pdf}}}

\column{0.5\textwidth}

\only<1>{\scalebox{0.35}{\includegraphics{fig2.pdf}}}


\only<2>{\scalebox{0.35}{\includegraphics{fig4.pdf}}}


\end{columns}


\end{frame}

\begin{frame}
\frametitle{Concave Up/ Convex}

\begin{defn} 
Suppose $f:[a,b]\rightarrow \Re$ is a \alert{twice} differentiable function.  If, for all $x \in [a,b]$ and $y \in [a,b]$ and $t \in (0,1)$
\begin{eqnarray}
f((1-t)x + t y) & < & (1-t) f(x) + t f(y) \nonumber 
\end{eqnarray} 
We say that $f$ is strictly \alert{concave up} or \alert{convex}.  Equivalently if $f^{''}(x)>0$ for all $x \in [a,b]$, we say that $f$ is strictly \alert{concave up}.  
\end{defn}

\end{frame}


\begin{frame}
\frametitle{Concave Up, Graphical Test}

$f(x) = e^{x}$, $[1, 4]$

\begin{center}
\only<1>{\scalebox{0.6}{\includegraphics{fig1.pdf}}}
\only<2>{\scalebox{0.6}{\includegraphics{Conv1.pdf}}}
\only<3>{\scalebox{0.6}{\includegraphics{Conv2.pdf}}}
\only<4>{\scalebox{0.6}{\includegraphics{Conv3.pdf}}}
\only<5>{\scalebox{0.6}{\includegraphics{Conv4.pdf}}}
\only<6>{\scalebox{0.6}{\includegraphics{Conv5.pdf}}}
\only<7>{\scalebox{0.6}{\includegraphics{Conv6.pdf}}}
\end{center}


\end{frame}

\begin{frame}
\frametitle{Concave Up, Second Derivative}


\begin{columns}[]

\column{0.6\textwidth}
\scalebox{0.6}{\includegraphics{fig1.pdf}}


\column{0.4\textwidth}
\pause 
\begin{eqnarray}
\invisible<1>{f(x) & = & e^{x} \nonumber \\} \pause 
\invisible<1-2>{f^{'}(x) & = & e^{x} \nonumber \\} \pause 
\invisible<1-3>{f^{''}(x) & = & e^{x} \nonumber } \pause 
\end{eqnarray}


\end{columns}
\invisible<1-4>{$e^{x} >0 $ for all $x \in [1, 4]$} 


\end{frame}



\begin{frame}
\frametitle{Concave Down}

\begin{defn} 
Suppose $f:[a,b]\rightarrow \Re$ is a \alert{twice} differentiable function.  If, for all $x \in [a,b]$ and $y \in [a,b]$ and $t \in (0,1)$
\begin{eqnarray}
f((1-t)x + t y) & > & (1-t) f(x) + t f(y) \nonumber 
\end{eqnarray} 
We say that $f$ is strictly \alert{concave down}.  Equivalently if $f^{''}(x)<0$ for all $x \in [a,b]$, we say that $f$ is strictly \alert{concave down}.  
\end{defn}


\end{frame}


\begin{frame}
\frametitle{Concave Down} 

\scalebox{0.6}{\includegraphics{Fig2.pdf}}


\begin{itemize}
\item[-] Show Concave down with graph test for $x \in [1,4]$
\item[-] Show concave down with second derivative test for $x \in [1,4]$
\end{itemize}


\end{frame}



\begin{frame}
\frametitle{Optimization}

\begin{thm} \alert{Extreme Value Theorem} Suppose $f: [a,b] \rightarrow \Re$ and that $f$ is continuous.  Then $f$ obtains its extreme value on $[a,b]$.  
\end{thm}

\begin{cor} Suppose $f: [a,b] \rightarrow \Re$,  that $f$ is continuous and differentiable, and that $f(a)$ nor $f(b)$ is the extreme value.  Then $f$ obtains its maximum on $(a,b)$ and if $f(x_{0})$ is the extreme value of $f$ $x_0 \in (a,b)$ then, $f^{'}(x_{0}) = 0$.  
\end{cor}

\end{frame}


\begin{frame}
\frametitle{Extrema on End Points}
\scalebox{0.5}{\includegraphics{Line1.pdf}}



\end{frame}



\begin{frame}
\frametitle{Maximum in Middle, Concave Down}

$f(x) = - x^2 + 5$.  


\scalebox{0.5}{\includegraphics{Rolles3.pdf}}



\end{frame}


\begin{frame}
\frametitle{Minimum in Interior, Concave Up}

$f(x) = x^2 + 9x + 9 $

\scalebox{0.5}{\includegraphics{Line3.pdf}}


\end{frame}



\begin{frame}
\frametitle{Local Optima}


$f(x) = \sin(x)$


\scalebox{0.5}{\includegraphics{Line4.pdf}}

\end{frame}


\begin{frame}
\frametitle{Inflection points}

$f(x) = x^3$

\scalebox{0.5}{\includegraphics{Line5.pdf}}

\end{frame}


\begin{frame}
\frametitle{Framework for Optimization}

\alert{Recipe} for optimization \pause 
\begin{itemize}
\invisible<1>{\item[-] Find $f^{'}(x)$. } \pause 
\invisible<1-2>{\item[-] Set $f^{'}(x) = 0$ and solve for $x$.  Call all $x_{0}$ such that $f^{'}(x_{0} ) = 0$ \alert{critical values}.} \pause 
\invisible<1-3>{\item[-] Find $f^{''}(x)$.  Evaluate at each $x_{0}$.  } \pause 
\begin{itemize}
\invisible<1-4>{\item[-] If $f^{''}(x)>0$, Concave up, \alert{local} minimum} \pause 
\invisible<1-5>{\item[-] If $f^{''}(x)<0$, Concave down, \alert{local} maximum} \pause 
\invisible<1-6>{\item[-] If $f^{''}(x) = 0$, No knowledge---local minimum, maximum, or inflection point} \pause 
\end{itemize}
\invisible<1-7>{\item[-] \alert{Check End Points!}} 
\end{itemize}

\end{frame}


\begin{frame}
\frametitle{Example 1: $f(x) = -x^2$,  $x \in [-3, 3] $}


\only<1>{\scalebox{0.5}{\includegraphics{OptExamp1.pdf}}} 


\only<2->{
	\invisible<1-2>{1) Critical Value:

	\begin{eqnarray}
	f^{'}(x) & = & - 2 x \nonumber \\
	0 & = & - 2 x^{*} \nonumber \\
	x^{*} & = & 0 \nonumber 
	\end{eqnarray}}

	\invisible<1-3>{2) Second Derivative: 

	\begin{eqnarray}
	f^{'}(x) & = & - 2x \nonumber \\
	f^{''}(x)  & = & - 2 \nonumber 
	\end{eqnarray}
	$f^{''}(x)< 0$, local maximum }
}





\pause \pause \pause \pause

\end{frame}


\begin{frame}
\frametitle{Example 1: $f(x) = -x^2$,  $x \in [-3, 3] $}

	3) Check end points
	\begin{eqnarray}
	f(0) & = & - 0 ^2 = 0 \nonumber \\
	f(-3) & = & - (-3)^2 = -9 \nonumber \\
	f(3) & = & - (3)^2 = -9 \nonumber 
	\end{eqnarray}


\end{frame}



\begin{frame}
\frametitle{Example 2: $f(x) = x^3$, $x \in [-3, 3] $ }

\only<1>{\scalebox{0.5}{\includegraphics{OptExamp2.pdf}}}


\only<2->{
	
1) Critical Values: 
\begin{eqnarray}
f^{'}(x) & = & 3x^2 \nonumber \\
0 & = & 3 (x^{*})^{2} \nonumber \\
x^{*} & = & 0 \nonumber 
\end{eqnarray}

\invisible<1-2>{2) Second Derivative: 

\begin{eqnarray}
f^{''}(x) & = & 6 x \nonumber \\
f^{''}(0) & = &  0 \nonumber 
\end{eqnarray}
\alert{No information}
}

}
\pause \pause 



\end{frame}

\begin{frame}
\frametitle{Example 2: $f(x) = x^3$, $x \in [-3, 3] $}


3) Check End Points: 

\begin{eqnarray}
f(0) & = & 0^3  = 0 \nonumber \\
f(-3) & = & -3^3 = -27 \nonumber \\
f(3) & = & 3^3 = 27 \nonumber 
\end{eqnarray}

Neither maximum nor minimum, \alert{saddle point} 










\end{frame}




%%add limits on the function here
\begin{frame}
\frametitle{Example 3: \alert{Spatial Model} }
A large literature in Congress supposes legislators and policies can be situated in \alert{policy space} \pause \\

\invisible<1>{Suppose legislator $i$ and policies $x, i \in \Re$. } \pause  \\
\invisible<1-2>{Define legislator $i$'s utility as, $U:\Re \rightarrow \Re$,} \pause 
\begin{eqnarray}
\invisible<1-3>{U_{i} (x) & = & - (x - \mu)^{2}  \nonumber \\} \pause 
\invisible<1-4>{U_{i}(x) & = & - x^2 +  2 x \mu  - \mu^2 \nonumber }\pause 
\end{eqnarray}
\invisible<1-5>{What is $i$'s optimal policy over the range $x \in [\mu- 2, \mu + 2]$?} \pause 
\begin{eqnarray}
\invisible<1-6>{U_{i}^{'} (x) & = & -2 (x - \mu) \nonumber \\} \pause 
\invisible<1-7>{0 & = & -2x^{*} + 2 \mu \nonumber \\} \pause 
\invisible<1-8>{x^{*} & = & \mu \nonumber } \pause 
\end{eqnarray}

\invisible<1-9>{\alert{Second Derivative Test}} \pause 
\begin{eqnarray}
\invisible<1-10>{U^{''}_{i}(x)  & = & -2 <0 \rightarrow \text{Concave Down} \nonumber } \pause 
\end{eqnarray}

\invisible<1-11>{We call $\alert{\mu}$ legislator $i$'s \alert{ideal point}} 

\end{frame} 

\begin{frame}
\frametitle{Example 3: \alert{Spatial Model} }

\begin{eqnarray}
U_{i}(\mu) & = & - (\mu - \mu)^2 = 0 \nonumber \\
U_{i}(\mu - 2) & = & - (\mu - 2 - \mu)^2 = -4 \nonumber \\
U_{i} (\mu + 2) & = & - (\mu + 2 - \mu)^2 = -4 \nonumber 
\end{eqnarray}

\alert{Maximize utility at $\mu$}



\end{frame}




\begin{frame}
\frametitle{Example 4: \alert{Maximum Likelihood Estimation}}

In statistics classes we'll learn about \alert{parameters} from \alert{data}.  \pause  \\
\invisible<1>{Here is an example \alert{likelihood} function: We want to find the \alert{Maximum likelihood estimate}} \pause \\
\begin{eqnarray}
\invisible<1-2>{f(\mu) & = & \prod_{i=1}^{N} \exp( \frac{-(Y_{i} - \mu)^2}{ 2}) \nonumber } \pause \\
\invisible<1-3>{   & = & \exp(- \frac{(Y_{1} - \mu)^2}{ 2}) \times \hdots \times \exp(- \frac{(Y_{N} - \mu)^2}{ 2})\nonumber } \pause \\
\invisible<1-4>{     & = & \exp( - \frac{\sum_{i=1}^{N} (Y_{i} - \mu)^2} {2}) \nonumber } \pause 
  \end{eqnarray}
  
\invisible<1-5>{ \begin{thm} Suppose $f:\Re \rightarrow (0, \infty)$.  If $x_{0}$ maximizes $f$, then $x_{0}$ maximizes $\log(f(x))$.  
 \end{thm} } 
 
   

\end{frame}

\begin{frame}
\frametitle{Example 4: \alert{Maximum LIkelihood Estimation}}

\pause 
\begin{eqnarray}
\invisible<1>{\log f(\mu) & = & \log \left( \exp( - \frac{\sum_{i=1}^{N} (Y_{i} - \mu)^2} {2}) \right) \nonumber \\} \pause 
	\invisible<1-2>{& = & - \frac{\sum_{i=1}^{N} (Y_{i} - \mu)^2} {2}) \nonumber \\} \pause 
		\invisible<1-3>{	& = & -\frac{1}{2} \left(\sum_{i=1}^{N} Y_{i}^2 - 2\mu \sum_{i=1}^{N} Y_{i} + N\times\mu^2 \right) \nonumber \\} \pause 
\invisible<1-4>{\frac{ \partial \log f(\mu) }{ \partial \mu } & = & 		-\frac{1}{2} \left( - 2\sum_{i=1}^{N} Y_{i} + 2 N \mu \right) \nonumber 		}
\end{eqnarray}




\end{frame}

\begin{frame}
\frametitle{Example 4: \alert{Maximum Likelihood Estimation} } 


\begin{eqnarray}
\invisible<1>{0 & = & -\frac{1}{2} \left( - \sum_{i=1}^{N} Y_{i} + 2 N \mu^{*} \right) \nonumber \\} \pause 
\invisible<1-2>{2 \sum_{i=1}^{N}Y_{i}  & = & 2 N \mu^{*} \nonumber \\} \pause 
\invisible<1-3>{\frac{\sum_{i=1}^{N}Y_{i}}{N} & = & \mu^{*} \nonumber \\} \pause 
\invisible<1-4>{\bar{Y} & =& \mu^{*} \nonumber } \pause 
\end{eqnarray}
\invisible<1-5>{\alert{Second Derivative Test} } \pause 
\begin{eqnarray}
\invisible<1-6>{f^{'}(\mu ) & = & -\frac{1}{2} \left( - 2\sum_{i=1}^{N} Y_{i} + 2 N \mu \right)  \nonumber \\} \pause 
\invisible<1-7>{f^{''}(\mu ) & = & -N \nonumber } \pause 
\end{eqnarray}
\invisible<1-8>{$-N<0$, \alert{concave down}, maximum(!!)} 

\end{frame}

\begin{frame}
\frametitle{Example 5: IR Bargaining (from Jim Fearon, Part 1) } 

\alert{Countries} fight wars, usually to get stuff.  \pause 

\begin{itemize}
\invisible<1>{\item[-] Suppose two countries $1,2$ are fighting for something they value at $v$.} \pause 
\invisible<1-2>{\item[-] Each country decides to invest $a_{1} \in [0,1]$ and $a_{2} \in [0,1]$.  } \pause 
\invisible<1-3>{\item[-] The probability of country $1$ winning the war is } \pause 
\begin{eqnarray}
\invisible<1-4>{p(a_{1}, a_{2} ) & = & \frac{a_{1}^{n} }{a_{1}^{n} + a_{2}^{n} } \nonumber } \pause 
\end{eqnarray}
\invisible<1-5>{\item[-] Country $1$'s utility is given by } \pause 
\begin{eqnarray}
\invisible<1-6>{U_{1}(a_{1} ) & = & \underbrace{1- a_{1}}_{\text{cost}} + \underbrace{p(a_{1}, a_{2})v}_{\text{Expected Benefit}} \nonumber \\
				& = & 1 - a_{1} + \frac{a_{1}^{n} }{a_{1}^{n} + a_{2}^{n} }  v \nonumber } \pause 
\end{eqnarray}
\invisible<1-7>{\item[-] Suppose country $2$ selected value $x$.  What should country $1$ invest to maximize utility?  } 
\end{itemize}



\end{frame}

\begin{frame}
\frametitle{Example 5: IR Bargaining (from Jim Fearon, Part 1) } 


\scalebox{0.5}{\includegraphics{Fearon1.pdf} } 


\end{frame}



\begin{frame}
\frametitle{Example 5: IR War (from Jim Fearon, Part 1) } 



\begin{eqnarray}
\frac{\partial U_{1} (a_{1}) }{\partial a_{1}} & = &  - 1 + \frac{n a_{1}^{n-1} (a_{1}^{n} + x^{n} ) - (n a_{1}^{n-1} a_{1}^{n}) }{(a_{1}^{n} + x^{n} )^2} v \nonumber \\
& =& - 1 + \frac{n a_{1}^{n-1} x^{n} }{(a_{1}^{n} + x^{n} )^2} v \nonumber 
\end{eqnarray}
Set $n = 1$ (for simplicity)
\begin{eqnarray}
0 & = &  -1  + \frac{x}{(a_{1} + x)^2}v \nonumber \\
a_{1}^{*} & = & \sqrt{v}\sqrt{x} - x \nonumber \\
\end{eqnarray}
Second derivative!
\begin{eqnarray}
U_{1}^{''} (a_{1})& = & \frac{- 2 v x  }{(a_{1} + x)^{3} } \nonumber
\end{eqnarray}


\end{frame}

\begin{frame}
\frametitle{Example 5: IR Bargaining (from Jim Fearon, Part 1) } 

One more---check endpoints
\begin{eqnarray}
a_{1}^{*} & = & 0, \text{ if } \sqrt{v}\sqrt{x} - x < 0 \nonumber \\
a_{1}^{*} & = & 0, \text{ if } \sqrt{v}< \sqrt{x} \nonumber \\
a_{1}^{*} & = & \sqrt{v}\sqrt{x} - x \text{ otherwise }\nonumber 
\end{eqnarray}


\end{frame}


\begin{frame}
\frametitle{Optimization \alert{Challenge} Problem}



\begin{itemize}
\item[-] Suppose a candidate is attempting to mobilize voters.  Suppose that for each investment of $x \in [0, \infty)$ the candidate receives
return of $x^{1/2}$, but incurs cost of $a x$.  So, candidate utility is, 
\begin{eqnarray}
U_{i} = x^{1/2}  - a x \nonumber 
\end{eqnarray}
\item[] What is the optimal investment $x^{*}$?

\end{itemize}

\begin{center}
\scalebox{0.375}{\includegraphics{Chall1.pdf}}
\end{center}


\end{frame}


\begin{frame}
\frametitle{Computational Optimization Approaches}

\alert{Analytic} (Closed form)$\leadsto$ Often difficult, impractical, or unavailable  \pause \\

\invisible<1>{\alert{Computational}}\pause\invisible<1-2>{$\leadsto$ \alert{iterative} algorithm that converges to a solution (hopefully the right one!) } \pause 
\begin{itemize}
\invisible<1-3>{\item[-] Methods for optimization:} \pause 
\begin{itemize}
\invisible<1-4>{\item[-] \alert{Newton's method} and related methods} \pause 
\invisible<1-5>{\item[-] Gradient descent (ascent)} \pause 
\invisible<1-6>{\item[-] Expectation Maximization} \pause 
\invisible<1-7>{\item[-] Genetic Optimization} \pause 
\invisible<1-8>{\item[-] Branch and Bound ...} 
\end{itemize}
\end{itemize}
\end{frame}


\begin{frame}
\frametitle{Newton-Raphson Method}

Iterative procedure to find a \alert{root} \\ \pause 
\invisible<1>{Often solving for $x$ when $f(x) = 0$ is hard$\leadsto$ complicated function} \pause \\
\invisible<1-2>{Solving for $x$ when $f(x)$ is linear$\leadsto$ easy} \pause \\
\invisible<1-3>{Approximate with \alert{tangent line}, iteratively update}


%%put an example here that shows how this might work

%\scalebox{0.5}{\includegraphics{TangentExample.pdf}}

%use tangent line to find root
\end{frame}



\begin{frame}
\frametitle{Tangent Line}


\only<1>{\scalebox{0.6}{\includegraphics{TanLine26.pdf}}}
\only<2>{\scalebox{0.6}{\includegraphics{TanLine1.pdf}}}
\only<3>{\scalebox{0.6}{\includegraphics{TanLine2.pdf}}}
\only<4>{\scalebox{0.6}{\includegraphics{TanLine3.pdf}}}
\only<5>{\scalebox{0.6}{\includegraphics{TanLine4.pdf}}}
\only<6>{\scalebox{0.6}{\includegraphics{TanLine5.pdf}}}
\only<7>{\scalebox{0.6}{\includegraphics{TanLine6.pdf}}}
\only<8>{\scalebox{0.6}{\includegraphics{TanLine7.pdf}}}
\only<9>{\scalebox{0.6}{\includegraphics{TanLine8.pdf}}}
\only<10>{\scalebox{0.6}{\includegraphics{TanLine9.pdf}}}
\only<11>{\scalebox{0.6}{\includegraphics{TanLine10.pdf}}}
\only<12>{\scalebox{0.6}{\includegraphics{TanLine11.pdf}}}
\only<13>{\scalebox{0.6}{\includegraphics{TanLine12.pdf}}}
\only<14>{\scalebox{0.6}{\includegraphics{TanLine13.pdf}}}
\only<15>{\scalebox{0.6}{\includegraphics{TanLine14.pdf}}}
\only<16>{\scalebox{0.6}{\includegraphics{TanLine15.pdf}}}
\only<17>{\scalebox{0.6}{\includegraphics{TanLine16.pdf}}}
\only<18>{\scalebox{0.6}{\includegraphics{TanLine17.pdf}}}
\only<19>{\scalebox{0.6}{\includegraphics{TanLine18.pdf}}}
\only<20>{\scalebox{0.6}{\includegraphics{TanLine19.pdf}}}
\only<21>{\scalebox{0.6}{\includegraphics{TanLine20.pdf}}}
\only<22>{\scalebox{0.6}{\includegraphics{TanLine21.pdf}}}
\only<23>{\scalebox{0.6}{\includegraphics{TanLine22.pdf}}}
\only<24>{\scalebox{0.6}{\includegraphics{TanLine23.pdf}}}
\only<25>{\scalebox{0.6}{\includegraphics{TanLine24.pdf}}}
\only<26>{\scalebox{0.6}{\includegraphics{TanLine25.pdf}}}


\end{frame}



\begin{frame}
\frametitle{Tangent Line}

Formula for Tangent line at $x_{0}$: 


\begin{eqnarray}
\invisible<1>{g(x) =  \alert<4>{f^{'}(x_{0})} \alert<5>{(x - x_{0} )} + \alert<3>{f(x_{0} )} \nonumber }
\end{eqnarray}


\invisible<1-5>{We'll use formula for tangent line to generate updates}


\pause \pause \pause \pause 


\end{frame}






\begin{frame}
\frametitle{Newton-Raphson Method}

Suppose we have some initial guess $x_{0}$.  We're going to approximate $f^{'}(x)$ with the tangent line to generate a new guess


 \pause 

\begin{eqnarray}
\invisible<1>{g(x) & = & f^{''}(x_{0})(x - x_{0} ) + f^{'}(x_{0} ) \nonumber \\}
\invisible<1-2>{0 & = & f^{''}(x_{0}) (x_{1} - x_{0}) + f^{'}(x_{0} ) \nonumber \\} %because at the root the function has to equal zero, so we're moving closer to it.
\invisible<1-3>{x_{1} & = &  x_{0} - \frac{f^{'}(x_{0}) }{f^{''}(x_{0})} \nonumber }
\end{eqnarray}

\invisible<1-4>{Perform iteratively until change in $|x_{t+1} - x_{t}|<$ threshold }

\end{frame}



\begin{frame}
\frametitle{Example Function }

$f(x) = x^3 + 2x^2 - 1$
find $x$ that maximizes $f(x) $ with $x \in [-3, 0]$


\scalebox{0.5}{\includegraphics{NRExample.pdf}}



\end{frame}



\begin{frame}

\begin{eqnarray}
f^{'}(x) & = & 3 x^2 + 4 x \nonumber \\
f^{''}(x) & = & 6x + 4 \nonumber 
\end{eqnarray}

Suppose we have guess $x_{t}$ then the next step is:
\begin{eqnarray}
x_{t+1} & = & x_t - \frac{3 x_{t}^2 + 4 x_{t}}{6 x_{t} + 4} \nonumber
\end{eqnarray}


\end{frame}


\begin{frame}
\frametitle<7->{$x^{*} = -1.3333$}

\only<1>{\scalebox{0.5}{\includegraphics{NRExamp1.pdf}}} 
\only<2>{\scalebox{0.5}{\includegraphics{NRExamp2.pdf}}}
\only<3>{\scalebox{0.5}{\includegraphics{NRExamp3.pdf}}}
\only<4>{\scalebox{0.5}{\includegraphics{NRExamp4.pdf}}}
\only<5>{\scalebox{0.5}{\includegraphics{NRExamp5.pdf}}}
\only<6>{\scalebox{0.5}{\includegraphics{NRExamp6.pdf}}}
\only<7>{\scalebox{0.5}{\includegraphics{NRExamp7.pdf}}}


\end{frame}


\begin{frame}
\frametitle{What is Happening with the Roots}

\only<1>{\scalebox{0.5}{\includegraphics{NewtRaphDeriv20.pdf}}}
\only<2>{\scalebox{0.5}{\includegraphics{NewtRaphDeriv1.pdf}}}
\only<3>{\scalebox{0.5}{\includegraphics{NewtRaphDeriv2.pdf}}}
\only<4>{\scalebox{0.5}{\includegraphics{NewtRaphDeriv3.pdf}}}
\only<5>{\scalebox{0.5}{\includegraphics{NewtRaphDeriv4.pdf}}}
\only<6>{\scalebox{0.5}{\includegraphics{NewtRaphDeriv5.pdf}}}
\only<7>{\scalebox{0.5}{\includegraphics{NewtRaphDeriv6.pdf}}}
\only<8>{\scalebox{0.5}{\includegraphics{NewtRaphDeriv7.pdf}}}
\only<9>{\scalebox{0.5}{\includegraphics{NewtRaphDeriv8.pdf}}}
\only<10>{\scalebox{0.5}{\includegraphics{NewtRaphDeriv9.pdf}}}
\only<11>{\scalebox{0.5}{\includegraphics{NewtRaphDeriv10.pdf}}}
\only<12>{\scalebox{0.5}{\includegraphics{NewtRaphDeriv11.pdf}}}
\only<13>{\scalebox{0.5}{\includegraphics{NewtRaphDeriv12.pdf}}}





\end{frame}



\begin{frame}


\only<1>{\scalebox{0.5}{\includegraphics{NRExamp8.pdf}}} 
\only<2>{\scalebox{0.5}{\includegraphics{NRExamp9.pdf}}}
\only<3>{\scalebox{0.5}{\includegraphics{NRExamp10.pdf}}}
\only<4>{\scalebox{0.5}{\includegraphics{NRExamp11.pdf}}}
\only<5>{\scalebox{0.5}{\includegraphics{NRExamp12.pdf}}}
\only<6>{\scalebox{0.5}{\includegraphics{NRExamp13.pdf}}}
\only<7>{\scalebox{0.5}{\includegraphics{NRExamp14.pdf}}} 
\only<8>{\scalebox{0.5}{\includegraphics{NRExamp15.pdf}}}
\only<9>{\scalebox{0.5}{\includegraphics{NRExamp16.pdf}}}
\only<10>{\scalebox{0.5}{\includegraphics{NRExamp17.pdf}}}
\only<11>{\scalebox{0.5}{\includegraphics{NRExamp18.pdf}}}
\only<12>{\scalebox{0.5}{\includegraphics{NRExamp19.pdf}}}
\only<13>{{\tt To the R Code!}}
\end{frame}








\begin{frame}
\frametitle{Today/Tomorrow}

\begin{itemize}
\item[-] A Framework for optimization
\begin{itemize}
\item[-] Analytic: pencil and paper math
\item[-] Computational: iterative algorithm that aids in solution
\end{itemize}
\item[-] Integration: antidifferentation/area finding
\end{itemize}

\end{frame}






\end{document}

